\documentclass{article}

\usepackage{amsmath}

\title{How gasbeam works}
\author{David Roundy}
\begin{document}
\maketitle This is my explanation based on my reading both of my notes
and of the program \verb!Dgt.cpp! found in the directory
\verb!Foggo Gasbeam!.  I'm writing this for Morty to understand (and
be able to edit), \emph{not paper-quality writing!}

\section{Purpose}
The point of this program is to compute the profile of a gas beam when
the pressure is sufficiently high that collisions within the gas are
significant.  In this case, the density of the gas within the tube is
an unknown, but assumed to be a linear function of $z$.  We are of
course wanting to find the angular distribution of the gas exiting the
tube.


\section{Overall strategy}
The code makes one huge simplification relative to a straightforward
Monte Carlo simulation.  \emph{We only simulate the trajectory of an
  atom after its last ``hard'' collision!} By \emph{hard} collision, I
mean a collision with a wall or a ``hard-sphere'' collision with
another atom.  These collisions effectively randomize the direction of
the atom, erasing any effect of previous collisions.  We do
incorporate ``soft sphere'' collisions, which have only a small effect
on the direction of the atom.

Once we have simulated enough trajectories (given the linear density
profile), we can predict the flow rate of the gas.  We then adjust the
density profile to match the desired value.  However, since the linear
density function has two parameters (an ``exit density'' and a
gradient), it's a bit more complicated.
\begin{enumerate}
\item Given a gas density profile along the tube, we simulate an
  ensemble of final trajectories.  We collect appropriate statistics.
\item From these statistics, we must compute flow rate and correct the
  density profile.
\end{enumerate}

\subsection{Simulating a atom trajectory}
When we simulate a trajectory, we have atoms always travel with the
mean speed of the atom in the gas.  This is obviously not true, but
shouldn't bias the results.

We begin an atom's trajectory immediately after a collision, which
could either be a collision with a wall, a collision with the back of
the tube (which is somewhat artificial, and could be omitted if the
tube were taken to be long enough) or a collision with another
atom.  The rate of collisions with a given area $dA$ of the
wall is equal to
\begin{align}
  dP_{\text{wall}} &= nvdA
\end{align}
where $n$ is the number density, $v$ is the mean speed, and $dA$ is
the area of the portion of wall.  We include a ``fake'' wall at the
back of the tube.  The rate of atom-atom collisions in a small volume
of gas is given by
\begin{align}
  dP_{\text{hard-sphere}} &= n^2\sigma_{HS}v dV
\end{align}
where $\sigma_{HS}$ is the hard-sphere cross-section and $dV$ is the
small volume. You can think of this as coming from the total area
available to be colided with being $dA_{HS} = n\sigma_{HS}dV$.

We randomly sample final collisions in proportion to these rates.  For
each final collision, we then pick an initial direction.

\subsubsection{Picking the direction}
In the case of a hard-sphere collision, the direction is random.
Technically, the velocity is random after a hard-sphere scattering
only in the center-of-mass reference frame, but we assume that the
center-of-mass reference frame is itself random.

For a wall scattering (assuming diffuse scattering), the probability
of scattering direction per solid angle is proportional to the
$\cos\theta$ where $\theta$ is the angle from the normal.

In practice, we do not consider any ``backwards-traveling'' states,
under the assumption that these states will not exit the tube before
undergoing another collision.

\subsubsection{Propogating the atom}
Once we have its initial velocity (after its final randomizing
collision), we just let the atom travel step by step while undergoing
small randomizing collisions.  We travel by a small distance $d\ell$
which is chosen to be no greater than half a hard-sphere radius and
furthermore to be small enough that the density changes by less than
10\%.  At each step, we scatter the direction of the atom by a change
in angle
\begin{align}
  \Delta\theta &= \sqrt{n\sigma_{SS}d\ell}
\end{align}
where $\sigma_{SS}$ is called \verb!sigmaExtra! in the code, and I
have not yet figured out what it is.  Scattering by a fixed angle is
somewhat hokey, but thanks to the central limit theorem
this should not affect the answer so long as our RMS change in angle
is correctly chosen, and we make sure we are adding together many
small scatters.  If in the process of this propogation hits a wall or
exits the tube then we are done.  We ignore the case where it hits a
wall.

If it exited the tube, then we add it to our histogram of detected
atoms.  However, while being propogated, it should attenuate by a
factor of $e^{-n\sigma_{HS}d\ell}$ at each step.  This is to account
for the probability that it undergoes a hard sphere collision.

\end{document}
