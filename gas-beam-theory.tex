\documentclass{article}

\usepackage{color}
\usepackage{amsmath}
\newcommand\fixme[1]{\textcolor{red}{\sc #1}}

\title{How gasbeam works}
\author{David Roundy}
\begin{document}
\maketitle This is my explanation based on my reading both of my notes
and of the program \verb!Dgt.cpp! found in the directory
\verb!Foggo Gasbeam!.  I'm writing this for Morty to understand (and
be able to edit), \emph{not paper-quality writing!}

\section{Purpose}
The point of this program is to compute the profile of a gas beam when
the pressure is sufficiently high that collisions within the gas are
significant.  In this case, the density of the gas within the tube is
an unknown, but assumed to be a linear function of $z$.  We are of
course wanting to find the angular distribution of the gas exiting the
tube.

\fixme{Add explanation why we choose flow rate rather than input pressure as a parameter.}

\section{Overall strategy}
The code makes one huge simplification relative to a straightforward
Monte Carlo simulation.  \emph{We only simulate the trajectory of an
  atom after its last collision with a wall!} Wall collisions effectively randomize the direction of
the atom, erasing any effect of previous collisions.

Once we have simulated enough trajectories (given the linear density
profile), we can predict the flow rate of the gas.  We then adjust the
density profile to match the desired value.  \fixme{add bit here about the relationship between exit density and gradient.}
\begin{enumerate}
\item Given a gas density profile along the tube, we simulate an
  ensemble of final trajectories.  We collect appropriate statistics.
\item From these statistics, we must compute flow rate and correct the
  exit density.
\end{enumerate}

\subsection{Simulating a atom trajectory}
When we simulate a trajectory, we have atoms always travel with the
mean speed of the atom in the gas.  This is obviously not true, but
shouldn't bias the results.

We begin an atom's trajectory immediately after a wall collision, which could be a collision with the back of
the tube (which is somewhat artificial, and could be omitted if the
tube were taken to be long enough).  The rate of collisions with a given area $dA$ of the
wall is equal to
\begin{align}
  dP_{\text{wall}} &= nvdA
\end{align}
where $n$ is the number density, $v$ is the mean speed, and $dA$ is
the area of the portion of wall.  We include a ``fake'' wall at the
back of the tube.

We randomly sample locations of final wall collisions in proportion to these rates.  For
each final collision, we then pick an initial direction.

\subsubsection{Picking the direction}
For diffuse scattering off a wall, the probability
of scattering direction per solid angle is proportional to the
$\cos\theta$ where $\theta$ is the angle from the normal.

In practice, we omit all ``backwards-traveling'' states (in terms of sign of the $\hat z$ component of their velocity),
under the assumption that these states will not exit the tube before
undergoing another collision.

\subsubsection{Propagating the atom}
Once we have its initial velocity (after its final randomizing
collision), we just let the atom travel step by step while undergoing
randomizing collisions.  We assume that each collision has only a small effect on the direction of the atom.  This may seem odd given that collisions are often viewed as hard-sphere collisions, and hard-sphere collisions result in random directions in the center of mass frame.  But the center of mass frame is biased by the velocity of the scattering particle, resulting in a strongly forward-peaked post-collision velocity distribution.  Given that each collision has a small effect, we apply the central limit theorem to justify picking a fixed change of direction.

We let the atom travel by a small distance $d\ell$
which is chosen to be no greater than half the radius of the tube and
furthermore to be small enough that the density changes by less than
10\%.  At each step, we scatter the direction of the atom by a change
in angle
\begin{align}
  \Delta\theta &= \sqrt{n\Sigma d\ell}
\end{align}
where $\Sigma$ is defined by
\begin{align}
    \Sigma \equiv \int \theta^2 \frac{d\sigma}{d\Omega} d\Omega
\end{align}
where $\theta$ is the change in angle measured in the lab frame, and $\frac{d\sigma}{d\Omega}$ is the differential cross section.  We estimate $\Sigma$ from the hard-sphere cross section of a species by the expression
\begin{align}
    \Sigma = (1.354)^2\sigma_{HS}
\end{align}
where the numerical factor was found from a Monte Carlo simulation simulation of hard-sphere collisions with an atom moving in a random direction.
If in the process of this propagation the atom hits a wall or
exits the tube then we are done.  We ignore the case where it hits a
wall.

If the atom exited the tube, then we add it to our histogram of detected
atoms.

\subsubsection{Computing the flow rate}
The gas density at the exit of the tube ($n_0$) is related to the derivative of the gas density $n'$ by
\begin{align}
    n_0 &= D n'
\end{align}
where $D$ is the diameter of the tube.  \fixme{I am sure there is a good reason for this... but not sure what it is.}  Thus given the exit density and the above simulation, we can compute the flow rate through the tube.  To simulate the profile for a given flow rate, an initial guess is made for the exit density, and by running successive simulations computing the flow rate, we can solve for the exit density that leads to the desired flow rate.

\end{document}
